\subsection{Konceptforslag til mekaniske dele} \label{Konceptforlag - mekaniske dele}

\begin{figure}[H]
    \centering
    \begin{subfigure}[b]{0.34\textwidth}
        %centering
        \includegraphics[width=\textwidth]
        {Sections/5 Konceptgenerering/Media/1.Løsning.png}
        \caption{Konceptforslag 1 \protect\lillacirc}
        \label{fig:Konceptforslag 1}
    \end{subfigure}
    \begin{subfigure}[b]{0.32\textwidth}
        %centering
        \includegraphics[width=\textwidth]
        {Sections/5 Konceptgenerering/Media/2.Løsning.png}
        \caption{Konceptforslag 2 \protect\bluebox}
        \label{fig:Konceptforslag 2}
    \end{subfigure}
    \begin{subfigure}[b]{0.24\textwidth}
        %centering
        \includegraphics[width=\textwidth]
        {Sections/5 Konceptgenerering/Media/3.Løsning.png}
        \caption{Konceptforslag 3 \protect\cyanbox}
        \label{fig:Konceptforslag 3}
    \end{subfigure}
    \caption{Konceptforslag 1 til 3}
\end{figure} \plainbreak{-0.5}

\textbf{Konceptforslag 1  \protect\lillacirc}  bevæger sig lineært i tre dimensioner og benytter sig af en mekanisme der afsætter dråber, som tager udgangspunkt i en inkjet. Til at fastholde emnet, som skal prikkes, fastspændes det ved brug af sug. Her vil der skabes undertryk under emnet, som vil resultere i at emnet er fastspændt. Sammen med sug vil emnet understøttes af en fast plade under emnet.

\textbf{Konceptforslag 2 \protect\bluebox} 
bevæger sig på samme måde som koncept 1. Koncept 2 bruger som koncept 1 også dråbemekanismen. Denne iteration er unik i fastholdelsen, hvor der bruges pres på emnet fra siderne af, til at holde emnet fast. Her er, fremfor understøttelse i en plade, understøttelse i punkter, som er hvor emnet skal stabiliseres over flere kontaktpunkter.


\textbf{Konceptforslag 3 \protect\cyanbox} 
har sin bevægelse igennem en deltarobot, som trækker i sine arme for at flytte på hovedet over prikfladen. Der gøres også brug af mekanisme der afsætter dråber. Indspændingen laves ved at have en vakuumpose der er fyldt med små kugler, som bliver låst i position efter luften tages ud af posen. Hvis prikobjektet placeres på posen, vil de små kugler position låse objektets position. Dette koncept gør brug af pladen under emnet, som understøtning.

\begin{figure}[H]
    \centering
    \begin{subfigure}[b]{0.27\textwidth}
        %centering
        \includegraphics[width=\textwidth]
        {Sections/5 Konceptgenerering/Media/4.Løsning.png}
        \caption{Konceptforslag 4 \protect\blueangle}
        \label{fig:Konceptforslag 4}
    \end{subfigure}
    \begin{subfigure}[b]{0.28\textwidth}
        %centering
        \includegraphics[width=\textwidth]
        {Sections/5 Konceptgenerering/Media/5.Løsning.png}
        \caption{Konceptforslag 5 \protect\greenangle}
        \label{fig:Konceptforslag 5}
    \end{subfigure}
    \begin{subfigure}[b]{0.35\textwidth}
        %centering
        \includegraphics[width=\textwidth]
        {Sections/5 Konceptgenerering/Media/6.Løsning.png}
        \caption{Konceptforslag 6 \protect\gulangle}
        \label{fig:Konceptforslag 6}
    \end{subfigure}
    \caption{Konceptforslag 4 til 6}
\end{figure} \plainbreak{-0.5}

\textbf{Konceptforslag 4 \protect\blueangle} 
bevæger sig med udgangspunkt i en deltarobot, ligesom koncept 3. Koncept 4 gør brug af mekanismen der sætter dråber. Emnet vil ligge på en rug overflade, så emnet ikke bevæger sig. Derudover vil emnet blive understøttet af en plade under emnet.


\textbf{Konceptforslag 5 \protect\greenangle} 
placerer emnet på en drejeskive, så prikværktøjet kun skal bevæge sig fra side til side (x-retningen), op og ned (z-aksen) og rotere emnet om z-aksen. Prikværktøjet benytter mekanismen der afsætter dråber prikplacering. Emnet indspændes igennem posen med kugler, der sættes i vakuum og låser emnet. Emnet er understøttet af en plade.


\textbf{Konceptforslag 6 \protect\gulangle} 
bruger en flerleddet robotarm til sine bevægelser, som gør rotationer i led til en kontrolleret bevægelse til prikplacering. I dette koncept gøres der brug af tuschmetoden, som indebærer at føre et farvefyldt emne ned og berøre prikfladen, for at overføre farven. Konceptet gør brug af indspænding fra siden, og gør også brug af en plade under emnet til understøtning.

\begin{figure}[H]
    \centering
    \begin{subfigure}[b]{0.3\textwidth}
        %centering
        \includegraphics[width=\textwidth]
        {Sections/5 Konceptgenerering/Media/7.Løsning.png}
        \caption{Konceptforslag 7 \protect\orangeangle}
        \label{fig:Konceptforslag 7}
    \end{subfigure}
    \begin{subfigure}[b]{0.3\textwidth}
        %centering
        \includegraphics[width=\textwidth]
        {Sections/5 Konceptgenerering/Media/8.Løsning.png}
        \caption{Konceptforslag 8 \protect\pinkstar}
        \label{fig:Konceptforslag 8}
    \end{subfigure}
    \begin{subfigure}[b]{0.3\textwidth}
        %centering
        \includegraphics[width=\textwidth]
        {Sections/5 Konceptgenerering/Media/9.Løsning.png}
        \caption{Konceptforslag 9 \protect\redkant}
        \label{fig:Konceptforslag 9}
    \end{subfigure}
    \caption{Konceptforslag 7 til 9}
\end{figure} \plainbreak{-0.5}

\textbf{Konceptforslag 7 \protect\orangeangle} 
benytter en lineær bevægelse til bevægelse af prikværktøjet. Prikmetoden er som koncept 6 tuschmetoden. Der gøres her brug af en indspænding fra siderne. Der bruges en plade under emnet til at understøtte emnet.


\textbf{Konceptforslag 8 \protect\pinkstar}
bevæger sig i form af en deltarobots armjustering. Konceptet skal sætte prikker med tuschmetoden som koncept 6 og 7. Indspændingen kommer af vakuumpose med fyld, der låser emnets placering. Der bruges også her en plade under emnet til understøttelse.


\textbf{Konceptforslag 9 \protect\redkant} 
bevæger sig lineært i tre retninger for at justere prikværktøjet, samt benytter en prik metode, som en tusch. Emnet indspændes ved at danne et undertryk under emnet ved brug af sug, samt bliver understøttet af en plade under emnet.


