\chapter{Problemformulering}
Speckle patterns er essentielle ved brugen af DIC til måling af materialeegenskaber. Tilgængelige metoder til påføring af speckle patterns på overflader er enten manuelle og tidskrævende, eller laver prikker der ikke er optimale til DIC. Robotter anvendes blandt andet til automatisering og effektivisering i fremstillingsindustrien, hvor de indgår som led i produktionen. På baggrund heraf er følgende problemformulering udformet:


\begin{displayquote} 
\large
\centering
    \textit{Hvordan designes en robot til påføring af speckle patterns i makroskala, på et plan i 2D, der kan anvendes ved Digital Image Correlation?}
\end{displayquote}


\section{Afgrænsning} \label{Afgrænsning}
Projektet afgrænses ud fra et ønske om at fokusere på studierelavanter afspekter af løsningsudviklingen. Det er valgt, at udelade design og udvikling af elektriske kredsløb og styringssystemer, samt programmering af robotten fra projektets omfang. Selvom der afgrænses fra udviklingen af de elektroniske aspekter af løsningen tages der stadig højde for, at de skal være inkluderet. Eksempelvis placering af styringsenheden. 

Det er bekendt, at farvemidlet til påføring af speckle patterns påvirker materialer på forskellige måder. Farvemidlers kemiske opbygning og fremstilling er udenfor studiets fagområde, og undersøges ikke i dette projekt.
%Dette valg sikrer, at fokusset udelukkende ligger på de mekaniske komponenter og de udfordringer, der er forbundet med deres design og funktion.

%Ved at holde projektet fokuseret på de mekaniske dele bliver omfanget både klart defineret og håndterbart. Denne afgrænsning giver mulighed for at foretage en dybdegående analyse og udvikling af de mekaniske systemer, uden at blive hindret af de elektriske og kemiske aspekter. Resultatet er en mere målrettet indsats, hvor der kan optimeres de mekaniske løsninger med den nødvendige præcision og kvalitet.

I dette projekt forstås 2D planer som kontinuerlige planer uden krumning.