\subsection{Anvendelsessteder} \label{Anvendelsessteder}
%DIC bruges på samtlige størrelser og materialer, derfor afgrænses der i dette afsnit til at få en konkret størrelse, der skal kunne arbejdes med.

Speckle patterns opdeles i denne rapport i to dele, mikro- og makroskala. I denne rapport defineres mikroskala som den størrelse, der ikke kan opfanges med det blotte øje, og makroskala er den størrelse, der kan fanges med det blotte øje. Der vælges at arbejde i makroskala, med prikker ned til \SI{0,1}{mm} i diameter, fordi det vurderes, at være 


% Dermed vil det være mest effektivt, at lave en løsning til makroskopisk skala. Dette korreleres til DIC, ved at finde den mindste prikdiameter, der kan observeres uden hjælpemidler. observeres prikker med en diameter ned til $\SI{0,1}{mm}$, dermed sættes den nedre grænse for prikstørrelser for løsningen til $\SI{0,1}{mm}$ i prikdiameter. 


% Da en robot skal være en løsning, vælges det også at den maksimale størrelse på arbejdsområdet, hvorpå et speckle pattern skal påføres, må være: $\SI{200}{mm} \times \SI{150}{mm} \times \SI{50}{mm}$. Den højeste fællesnævner af disse værdier er 50, som giver et forhold på 4:3. Dette er valgt, da det er et normalt billedformat på kameraer. Dette er gjort for at mindske pixel spild. Pixel spild kommer når billedformatet ikke stemmer overens med arbejdsområdets højde og bredde forhold. Dette sker fordi forholdet af billedet, ikke stemmer overens med forholdet af kameraet. Her vil der enten være pixels der er uden for kameraet, eller så skal kameraet zoome mere ud, og mindre zoom vil kræve større pixels.