\section{Robotter} \label{Robotter}

Robotter er et mekanisk eller mekatronisk system, der selvstændigt eller delvist selvstændigt kan udføre én eller flere handlinger. Systemet styres typisk af en programmerbar styreenhed, der enten kan følge forudbestemte instruktioner eller tilpasse sin adfærd baseret på sensorinput. Robotter anvendes ofte til at automatisere opgaver, hvor der er krav om gentagelse, præcision eller effektivitet, og hvor manuel udførelse kan være forbundet med variation eller upræcise resultater. En robot består grundlæggende af flere integrerede systemer, der sammen muliggør kontrol over bevægelser og handlinger. Det mekaniske system danner grundstrukturen og kan være enten stationært eller bevægeligt, afhængigt af hvordan det skal interagere med emnet. \parencite{TextbookofRobotics}

%På baggrund af afsnit \ref{Fremstilling af Speckle pattern}, vurderes det, at robotter med fordel kan anvendes til fremstilling af speckle patterns. Manuelle metoder til at påføre speckle patterns, som for eksempel brug af airbrush eller tusch, kan føre til variationer i kvaliteten af mønstret. Disse variationer kan have en negativ indvirkning på nøjagtigheden og reproducerbarheden af målinger i DIC. Ved at anvende en robot forventes det, at påføringsprocessen kan standardiseres, hvilket sikrer en mere ensartet kvalitet og dermed mere pålidelige måleresultater.

%En vigtig del af robotten er påføringsmekanismen, som kan variere alt efter metode. Den kan for eksempel være baseret på sprøjteteknikker som airbrush, trykbaserede metoder som stempling eller andre teknologier, der overfører et prædefineret mønster, eksempelvis via print eller midlertidige tatoveringer. \parencite{TextbookofRobotics}

For at opnå nøjagtige bevægelser indgår der typisk sensorer i robotløsninger, der kontroller, at den teoretiske bevægelse også er udført i virkeligheden. Disse sensorer kan være kameraer, afstandsmålere eller kraftmålere, som giver systemet mulighed for at opdage afvigelser. Sensorerne sender data tilbage til styreenheden, som kan være en mikrocontroller eller computer, der i realtid kan behandle dataen og foretager de nødvendige justeringer. Dette samarbejde mellem sensorer, styreenhed og mekaniske komponenter er afgørende for, at kunne udføre de ønskede bevægelser og handlinger med nøjagtighed. \parencite{BasicsofRobotics}

Hvilken type robot, der kan være velegnet til opgaven, afhænger af en række forskellige faktorer. Eksempelvis kan størrelsen og geometrien på de testemner, der arbejdes med, have indflydelse på, hvordan systemet bør være udformet og bevæge sig i forhold til emnet. Desuden kan der være forskelle i, hvordan forskellige løsninger håndterer bevægelse, positionering og interaktion med omgivelserne. Derfor er det relevant at undersøge, hvilke forskellige typer af robotter der findes, og hvordan de adskiller sig fra hinanden.