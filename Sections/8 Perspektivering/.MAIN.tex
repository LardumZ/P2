\chapter{Perspektivering} \label{Perspektivering}
I dette kapitel diskuteres projektets begrænsninger og dets videreudviklingspotentiale af robot til påføring af speckle patterns. Systemet er i sin nuværende form designet til flade, todimensionelle overflader, hvilket har været et praktisk udgangspunkt i udviklingsarbejdet. Ønsker man at anvende løsningen på mere komplekse, tredimensionelle geometrier, kræver det en række mekaniske og styringsmæssige tilpasninger. Her kunne man med fordel overveje at erstatte den lineære bevægelse med en robotarm, da denne kan give mere bevægelsesfrihed og mulighed for at orientere sig i forhold til objektets form. Robotarmen vil bedre kunne følge konturerne på uregelmæssige emner under prikplacering. 

Udover geometri spiller størrelsen af arbejdsområdet også en rolle for løsningens anvendelighed. Det nuværende system har et begrænset arbejdesområde på \SI{200}{mm} $\times$ \SI{150}{mm} hvilket gør det velegnet til mindre emner. Ved at opskalere systemets dimensioner, enten gennem forlængede bevægelsesbaner eller ved at udnytte en robotarms rækkevidde, kan man muliggøre behandling af større komponenter eller flere emner ad gangen.

Grundlaget for projektets krav er problemanalysen, og der er he r ikke taget højde for en specifik kunde. Det havde givet mere indsigt i reelle mangler og vægtninger, hvis kunder var blevet indblandet i processen, og ville muligvis have givet andre fokuspunkter. Produktet løser de krav der er stillet, det vides ikke om den løser de faktiske krav en kunde ville stille.

Et centralt aspekt, som har stort udviklingspotentiale, er systemets vægt. I sin nuværende form anvendes primært stål og aluminiumsdele, hvilket bidrager til høj mekanisk stabilitet, samtidig med at øge den samlede masse betydeligt. Der er mulighed for at reducere vægten uden nødvendigvis skifte materiale. Gennem målrettet designoptimering, eksempelvis ved at anvende udskæringer, hulsystemer eller topologioptimerede komponenter, kan man reducere materialeforbruget i stål og aluminiumsdelene markant uden at gå på kompromis med stivhed og funktion. I ikke strukturelle områder kan lettere materialer som kompositter og 3D-printede plastdele anvendes, hvilket yderligere mindsker vægten. En lettere konstruktion vil reducere belastningen på bevægelige dele, hvilket kan føre til hurtigere bevægelser, lavere energiforbrug og mindre mekanisk slid. Vægtbesparelser er har både fordele i forhold til håndtering og fleksibilitet og er en kilde til øget energieffektivitet og længere levetid for systemets mekaniske dele. 

Afslutningsvis har rapporten vist at automatisering af speckle pattern påførring, ved brug af en robot, har potentiale til at producere speckle patterns med højere genskabelighed end nuværende løsninger. Der er behov for ændringer til designet, før det vurderes at have reel anvendleses potentiale. Ved optimering af doserings mekanisme, vægt, bevægelsesfrihed og systemfleksibilitet kan løsningen bringes tættere på praktisk og industriel anvendelse.

%Den bevægelsesmåde der er beskrevet i rapporten, ved at bevæge sig frem og tilbage over emnet i baner er ikke den mest optimale måde at placere prikker på. Det mest optimale ville være at have robottten til at bevæge sig fra prik til prik og derved minimere den distance som der skal bevæges. 

%- Vi skulle have snakket med nogle kunder


\begin{comment}
 - Speckle pattern på 3D objekter.\\
- Større arbejdsområde \\
- Robot arm i stedet for lineær bevægelse \\
- T-slots nedfræset i bundpladen til indspænding\\
- Kortere bolte til indspænding, så de ikke går så langt udover bundpladen, hvis det er et stort emne.\\
- Bruge bælter istedet, fordi vi nu har en motor der også skal devæges, hvilket også skaber usikkerhed. \\
-Punkter til vægtbesparelse\\
-Kamera placering\\
- Placering af lineær bevægelse ovenpå den nedserte del (fjerne sidder og toppen), men fravalgt pga. af tidsmangel\\
- Alle led skal smøres, aå vi kunne lige så godt have brugt en ball screw istedet\\
- Undvære følgestænger, fordi der ikke er behov for en sikkerhedsfaktor på 5


   En anden udviklingsmulighed knytter sig til intergration af avanceret sensorik. I den nuværende opstilling sker fåførigne ud fra forudefinerede positioner uden realtidsfeedback. 
\end{comment}

