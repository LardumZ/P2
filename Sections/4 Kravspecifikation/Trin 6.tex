\subsection{Trin 6 - Relationer mellem designspecifikationer og kundernes ønsker} \label{Trin 6}
I det 6. trin beskrives relationen mellem kundens ønsker og de udarbejdede designspecifikationer. Her er der givet henholdsvis 1 point for en lav relation $(\bigtriangledown)$, 3 point for en middel relation $(\bigcirc)$ og 9 point for en stærk relation $(\blackbigcirc)$.  Det efterstræbes at alle ønsker som minimum har én stærk relation til to designspecifikationer, som det ses i tabel 4.3 i trin 6. Herved sikres, at ønskerne er ligeligt repræsenteret i designspecifikationerne.

\begin{table}[H]
\centering
\caption{ Relation mellem ønsker og designspecifikationer. $\protect\blackbigcirc$ = stærk relation, $\bigcirc$ = moderat relation og $\bigtriangledown$ = svag relation.}
\label{fig: HOQ trin 6} 
\footnotesize
\def\arraystretch{2}
\begin{tabular}{|p{4.cm}|c|c|c|c|c|c|c|c|c|c|c|c|}
    \hline
    \cellcolor{aaublue} \textcolor{white}{\diagbox[height=6cm, width=4.4cm]{\raisebox{0.9\height}{\enspace \textbf{\large Ønsker}}}{\raisebox{-0.9\height}{\enspace \makecell{\textbf{\normalsize Design-} \\ \textbf{\normalsize Specifikationer}}}}} & \cellcolor{lightgray!20} \rotatebox{90}{\hspace{-2.9cm} 1. Prikstørrelse} & \cellcolor{lightgray!20} \rotatebox{90}{\hspace{-2.9cm} 2. Variation på prikstørrelse} & \cellcolor{lightgray!20} \rotatebox{90}{\hspace{-2.9cm} 3. Variation på prikplacering} & \cellcolor{lightgray!20} \rotatebox{90}{\hspace{-2.9cm} 4. Størrelse af arbejdsområde} & \cellcolor{lightgray!20} \rotatebox{90}{\hspace{-2.9cm} \makecell[l]{\addlinespace[-3pt] 5. FLytning af emnet under fremstilling \\ \quad af speckle pattern \vspace{-3pt} }} & \cellcolor{lightgray!20} \rotatebox{90}{\hspace{-2.9cm} \makecell[l]{\addlinespace[-3pt] 6. Antal forskellige farvemidler til \\ \quad speckle pattern \vspace{-3pt}}} & \cellcolor{lightgray!20} \rotatebox{90}{\hspace{-2.9cm} \makecell[l]{\addlinespace[-3pt] 7. Kontrast mellem prik og baggrund \vspace{-3pt} }} & \cellcolor{lightgray!20} \rotatebox{90}{\hspace{-2.9cm} 8. Tidsforbrug på prikplacering} &\cellcolor{lightgray!20} \rotatebox{90}{\hspace{-2.9cm} 9. Brugerinvolvering under process} & \cellcolor{lightgray!20} \rotatebox{90}{\hspace{-2.9cm} 10. Tid brugt på opsætning} & \cellcolor{lightgray!20} \rotatebox{90}{\hspace{-2.9cm} \makecell[l]{\addlinespace[-3pt] 11. Antal værktøj til at samle hele \\ \quad produktet }} & \cellcolor{lightgray!20} \rotatebox{90}{\hspace{-2.9cm} \makecell[l]{\addlinespace[-3pt] 12. Antal specialværktøj til at samle \\ \quad hele produktet \vspace{-3pt} }}  \\
    \hline
     \cellcolor{lightgray!20} \makecell[l]{\addlinespace[0pt] 1. Producere et godt \\ \quad speckle pattern \vspace{0pt}} & $\bigcirc$ & $\blackbigcirc$ & $\blackbigcirc$ & & $\blackbigcirc$  &  & $\bigcirc$ &  & & & & \\
    \hline
    \cellcolor{lightgray!20} 2. Genskabelighed & $\bigtriangledown$& $\blackbigcirc$ & $\blackbigcirc$ & & $\blackbigcirc$& & & & & &&\\
    \hline
    \cellcolor{lightgray!20} \makecell[l]{\addlinespace[0pt] 3. Håndtere forskellige \\ \quad objekt størrelser og former} \vspace{-10pt}  & $\blackbigcirc$& & & $\blackbigcirc$ &$\bigcirc$& & &$\bigcirc$& & & &\\
    \hline
    \cellcolor{lightgray!20} \makecell[l]{\addlinespace[0pt] 4. Håndtere forskellige \\ \quad materialer} \vspace{0pt} & & & & & $\bigcirc$ & $\blackbigcirc$& $\blackbigcirc$ & & & & & \\
    \hline
    \cellcolor{lightgray!20} \makecell[l]{\addlinespace[0pt]  5. Hurtig fremstilling af \\ \quad speckle pattern} \vspace{0pt} & $\bigcirc$& $\bigtriangledown$&$\bigtriangledown$& $\bigcirc$ & & &  & $\blackbigcirc$ & $\bigcirc$ & $\blackbigcirc$& & \\
    \hline
    \cellcolor{lightgray!20} 6. Brugervenlighed && & & & & & & & $\bigtriangledown$& $\bigcirc$ & $\blackbigcirc$ & $\blackbigcirc$ \\
    \hline
    \cellcolor{lightgray!20} 7. Lang levetid & & & & & & $\bigcirc$ & & $\bigtriangledown$& & &$\blackbigcirc$& $\bigcirc$ \\
    \hline
    \cellcolor{lightgray!20} 8. Lav arbejdsbyrde & & & & & & & & & $\blackbigcirc$ & $\blackbigcirc$ & $\bigcirc$& $\bigcirc$\\
    \hline
    \cellcolor{lightgray!20}  9. Lav førstegangspris&$\bigcirc$& $\bigcirc$& $\bigcirc$& $\bigcirc$& &$\bigcirc$ & & $\blackbigcirc$& & & & $\blackbigcirc$ \\
    \hline
    \cellcolor{lightgray!20} 10. Lav brugspris& & & & & & $\blackbigcirc$ & & & $\bigcirc$& $\blackbigcirc$& &  \\
    \hline
\end{tabular}
\end{table} \plainbreak{-0.5}

Relationer mellem ønsker og designspecifkationer kan være positive, såvel som negative. Designspecifikation 2 og 3 for eksempel har en positiv relation med ønsket om genskabelighed, men en negativ relation med ønsket om lav førstegangspris. For at skabe en høj genskabelighed skal produktet have en høj nøjagtighed. En forøgelse af nøjagtighed (og dermed lavere tolerancer) kommer ofte også med en højere pris. Dermed modstrider disse ønsker hinanden. Dette uddybes i afsnit \ref{Trin 8}.