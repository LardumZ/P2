\chapter{Konklusion} \label{Konklusion}
I dette projekt er der arbejdet med udvikling af et mekanisk design til en robot, der kan automatisere påføringen af speckle patterns i makroskala til brug i DIC, som beskrevet i problemformuleringen:
\begin{displayquote} 
\centering
    \textit{Hvordan designes en robot til påføring af speckle patterns i makroskala på et plan i 2D, der kan anvendes ved Digital Image Correlation?}
\end{displayquote}

Projektets primære formål har været at imødekomme udfordringerne ved manuelle påføringsmetoder, herunder varierende prikstørrelse, lav genskabelighed og høj arbejdsbyrde, ved at skabe en mere ensartet, effektiv og præcis løsning.

Gennem omfattende analyse af krav og eksisterende løsninger blev der udviklet en kartesisk robotløsning med fire frihedsgrader og en præcisionsdispenser (PeJV) til påføring af farve. Løsningen er i stand til at påføre prikker i størrelser 0,1 mm til 1,0 mm i diameter med en variation på under $\pm$ 0,05 mm. Desuden er nøjagtigheden af prikplaceringen beregnet til at være inden for $\pm$ 0,1 mm, hvilket ligger inden for kravene.

Effektiviteten af systemet blev vurderet ved at beregne den tid, det tager at dække en kvadratcentimeter. Løsningen er i stand til at påføre et speckle pattern med en hastighed på 15,1 sekunder pr. \(\text{cm}^2\), ifølge den kinematiske analyse. Dette er væsentligt hurtigere end de påkrævede 30 sekunder pr. \(\text{cm}^2\). Arbejdsområdet for løsningen er \SI{200}{mm} $\times$ \SI{150}{mm} $\times$ \SI{50}{mm}, hvilket gør den egnet til en lang række prøver og anvendelser. Derudover er løsningen designet sådan, at det giver mulighed for større objekter end arbejdsområdet, da der er åbne sider. Dette giver mulighed for at dække større emner i sektioner til hele emnet er dækket. 

Projektet har vist at, det er muligt at udvikle en robot, der påfører speckle patterns med høj præcision og lav variation, samtidig med at den reducerer den manuelle arbejdsbyrde betydeligt. Resultaterne peger på, at løsningen har potentiale til at blive et nyttigt værktøj i laboratoriemiljøer, hvor præcise værktøjer er nødvendige. Prikværktøjets pris er vurderet til omkring 89.000 DKK. Sammenlignet med andre konkurrerende løsninger er prisen højere (\ref{Problemanalyse}), løsningen er til gengæld automatisk og har mulighed for at genskabe et speckle pattern. Prisen vil kunne sænkes ved at finde et billigere prikværktøj og ved at producere flere dele af gangen.