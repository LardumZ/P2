This project explores the mechanical design of a robot intended for the automated application of speckle patterns used in Digital Image Correlation (DIC). DIC is a non-contact optical measurement technique used to determine strain by measuring deformation in a given material by comparing images of a specimen before and after loading and deformation. A key requirement for accurate DIC measurements is the application of high-quality speckle patterns' random dot distributions with specific contrast, density, and dot size criteria. Current manual methods for applying such patterns, including airbrushing, stamping and marker-based techniques, are time-consuming and prone to unwanted variability in dot placement and size. This project aims to improve pattern consistency and reduce manual workload by developing a robot capable of reliably applying speckle patterns in macroscale on 2D surfaces. The report presents a detailed analysis of speckle pattern requirements, existing application methods, and relevant robotic design considerations. Based on this, a mechanical concept is developed that prioritizes precision, repeatability and adaptability to various object sizes and materials. 

